%package list
\documentclass{article}
\usepackage[top=3cm, bottom=3cm, outer=3cm, inner=3cm]{geometry}
\usepackage{graphicx}
\usepackage{url}
\usepackage{hyperref}
\usepackage{array}
%\usepackage{multicol}
\newcolumntype{x}[1]{>{\centering\arraybackslash\hspace{0pt}}p{#1}}
\usepackage[numbers,super]{natbib}
\usepackage{pdfpages}
\usepackage{multirow}
\usepackage{cite}
\usepackage[normalem]{ulem}
% %% Own Packages %%
% \usepackage{csquotes}
% \MakeOuterQuote{"}
\usepackage{listings}



%%%%%%%%%%%%%%%%%%%%%%%%%%%%%%%%%%%%%%%%%%%%%%%%%%%%%%%%%%%%%%%%%%%%%%%%%%%%
%%%%%%%%%%%%%%%%%%%%%%%%%%%%%%%%%%%%%%%%%%%%%%%%%%%%%%%%%%%%%%%%%%%%%%%%%%%%
\newcommand{\csemail}{vmachacaa@ulasalle.edu.pe}
\newcommand{\csdocente}{MSc. Vicente Enrique Machaca Arceda}
\newcommand{\cscurso}{Construcción De Software}
\newcommand{\csuniversidad}{Universidad La Salle}
\newcommand{\csescuela}{Escuela Profesional de Ingeniería de Software}
\newcommand{\cspracnr}{04}
\newcommand{\cstema}{Base de datos}
%%%%%%%%%%%%%%%%%%%%%%%%%%%%%%%%%%%%%%%%%%%%%%%%%%%%%%%%%%%%%%%%%%%%%%%%%%%%
%%%%%%%%%%%%%%%%%%%%%%%%%%%%%%%%%%%%%%%%%%%%%%%%%%%%%%%%%%%%%%%%%%%%%%%%%%%%


\usepackage[english,spanish]{babel}
\usepackage[utf8]{inputenc}
\AtBeginDocument{\selectlanguage{spanish}}
\renewcommand{\figurename}{Figura}
\renewcommand{\refname}{Referencias}
\renewcommand{\tablename}{Tabla} %esto no funciona cuando se usa babel
\AtBeginDocument{%
	\renewcommand\tablename{Tabla}
}

\usepackage{fancyhdr}
\pagestyle{fancy}
\fancyhf{}
\setlength{\headheight}{30pt}
\renewcommand{\headrulewidth}{1pt}
\renewcommand{\footrulewidth}{1pt}
\fancyhead[L]{\raisebox{-0.2\height}{\includegraphics[width=4cm]{../../img/lasalle_black.pdf}}}
\fancyhead[C]{}
\fancyhead[R]{\fontsize{7}{7}\selectfont	\csuniversidad \\ \csescuela \\ \textbf{\cscurso} }
\fancyfoot[L]{MSc. Vicente Machaca}
\fancyfoot[C]{\cscurso}
\fancyfoot[R]{Página \thepage}







\begin{document}

\vspace*{10pt}

\begin{center}
    \fontsize{17}{17} \textbf{ Práctica \cspracnr}
\end{center}
%\centerline{\textbf{\underline{\Large Título: Informe de revisión del estado del arte}}}
%\vspace*{0.5cm}


\begin{table}[h]
    \begin{tabular}{|x{4.7cm}|x{4.8cm}|x{4.8cm}|}
        \hline
        \textbf{DOCENTE} & \textbf{CARRERA} & \textbf{CURSO} \\
        \hline
        \csdocente       & \csescuela       & \cscurso       \\
        \hline
    \end{tabular}
\end{table}


\begin{table}[h]
    \begin{tabular}{|x{4.7cm}|x{4.8cm}|x{4.8cm}|}
        \hline
        \textbf{PRÁCTICA} & \textbf{TEMA} & \textbf{DURACIÓN} \\
        \hline
        \cspracnr         & \cstema       & 2 horas           \\
        \hline
    \end{tabular}
\end{table}


\section{Datos de los estudiantes}
\begin{itemize}
    \item Grupo: 1
    \item Integrantes:
          \begin{itemize}
              \item Elvis Andre Cruces Gomez
              \item Yoshiro Milton Miranda Valdivia
              \item José Alfredo Pinto Villamar
          \end{itemize}
\end{itemize}





\section{Propuesta: PostgresSQL}\label{sec:Intro}
\subsection{Introducción}
En la actualidad los sistemas de información se manejan mediante las bases de
datos y se han convertido en elementos imprescindibles para nuestra vida
cotidiana. Dentro de este tema podemos encontrar las bases de datos
transaccionales que siguen 4 criterios muy importantes para su aplicación, el
conocido ACID (Atomicidad, consistencia, isolación, durabilidad). Entre las
transacciones que maneja las BD transaccionales, encontramos las On-Line
Transaction Processing (OLTP), enfocadas sobre todo en procesos de insert,
delete, update, luego encontramos el On-Line Analytical Processing (OLAP) que se
enfoca principalmente en hacer consultas. Existen 2 tipos de escalabilidad, el
Scale Up y Scale Out, las cuales son usadas en las bases de datos SQL y las
NoSQL, respectivamente. Dentro de las BD SQL se maneja el OLTP y en las NoSQL se
utiliza OLAP, donde se encuentra PostgreSQL, nuestra base de datos.

\subsection{Definición}
PostgreSQL, también conocido como Postgres, es un sistema de administración de
bases de datos relacionales gratuito y de código abierto que enfatiza la
extensibilidad y el cumplimiento de SQL.

\subsection{¿Por qué usar SQL?}
Dentro de los puntos a considerar para la realización del presente proyecto,
optamos por usar una base de datos SQL, por el principal hecho de que nosotros
vamos a manejar más el uso de registros para las asistencias y participaciones.
En cambio las NoSQL, están diseñadas más que todo para el tema de consultas y
para nuestro proyecto el tema de las consultas no es lo primordial. Otro punto
que resalta en el uso de las SQL, es que son menos vulnerables a fallas.

\subsection{Ventajas y desventajas}
\begin{itemize}
    \item Ventajas: Opensource y gratis. pgAdmin para el gestor ``fácil e intuitiva''.
        Extensibilidad (Python).
    \item Desventajas: La sintaxis de sus comandos puede llegar a
        ser poco intuitiva si no hay previo conocimiento en SQL.
\end{itemize}

%\clearpage
%\bibliographystyle{apalike}
\bibliographystyle{IEEEtranN}
\nocite{*}
\bibliography{refs.bib}
%\bibliography{bibliography}

\section{Repositorio}\label{sec:Repositorio}
\begin{itemize}
    \item {\color{blue}\href{https://github.com/pintovillamar/software-construction/tree/main/pract04-latex}{Practica 4}}
\end{itemize}


\end{document}